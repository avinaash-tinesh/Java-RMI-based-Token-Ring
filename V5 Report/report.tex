\documentclass[11pt, a4paper]{report}
\usepackage[utf8]{inputenc}
\usepackage[margin=1.2 in]{geometry}
\usepackage{graphicx}
\usepackage[T1]{fontenc}
\usepackage{charter}
\usepackage{amsmath}
\usepackage{setspace}
\usepackage{subcaption}
\usepackage{listings}
\usepackage{color}

\graphicspath{{Assets/}}

\title{CSCU9V5 Assignment Report}
\author{Student no: 2519302}
\date{November 25\textsuperscript{th} 2018}
\pagenumbering{roman}
\begin{document}
\begin{titlepage}
\maketitle
\clearpage\thispagestyle{empty}
\end{titlepage}
\doublespacing
\setstretch{1}
\tableofcontents
\thispagestyle{empty}
\newpage
\singlespacing

\addcontentsline{toc}{section}{1. \quad Problem Description}
\section*{Problem Description}

This report will detail the implementation of a token passing ring node network in Java. This implementation uses RMI as the means of communicating with each node in the network and communication between each node is directed by a token passed through this network. 

The general description of the task is as such, we have a ringManager class that acts as a client which initializes a connection with the first node in the network and passes a token to that node. From there the ringMemberImpl class that acts as a server node in the network with the task of receiving the token it is passed, recording that transaction onto a file and releasing the token on to the next node in the network. Each node in the network is to record their transactions onto the same file each time, this is allowed since RMI makes blocking calls to each node.  This is done until some stopping condition is provided. Below is a diagram that outlines the network architecture for this task:

\begin{center}
todo
\end{center} 

\addcontentsline{toc}{section}{2. \quad Assumptions}
\section*{Assumptions}

In the process of implementing the solution to this task, a few assumptions were made. The assumptions made were as follows:
\begin{itemize}
\item[1] 
\end{itemize}

\addcontentsline{toc}{section}{3. \quad Solution implementation}
\section*{Solution implementation}

\newpage
\definecolor{my_magenta}{rgb}{0.8, 0.2, 0.4}
\definecolor{my_blue}{rgb}{0.33, 0.56, 0.93}
\definecolor{my_gray}{rgb}{0.5,0.5,0.5}
\lstset{ %
  backgroundcolor=\color{white},   % choose the background color
  basicstyle=\small\ttfamily,        % size of fonts used for the code
  breaklines=true,                 % automatic line breaking only at whitespace
  captionpos=b,                    % sets the caption-position to bottom
  commentstyle=\color{my_gray},    % comment style
  escapeinside={\%*}{*)},          % if you want to add LaTeX within your code
  keywordstyle=\color{my_magenta},       % keyword style
  stringstyle=\color{my_blue},     % string literal style
  showspaces=false,
  showtabs=false,
  showstringspaces=false,
  tabsize=2,					   % set tabsize in listing 
}

\addcontentsline{toc}{section}{4. \quad Code Listings}
\section*{Code Listings}
\addcontentsline{toc}{subsection}{4.1 \quad ringManager.java}
\subsection*{ringManager.java}
\lstinputlisting[language=Java]{ringManager.java}
\newpage
\addcontentsline{toc}{subsection}{4.2 \quad ringMemberImpl.java}
\subsection*{ringMemberImpl.java}
\lstinputlisting[language=Java]{ringMemberImpl.java}
\newpage
\addcontentsline{toc}{subsection}{4.3 \quad ringMember.java}
\subsection*{ringMember.java}
\lstinputlisting[language=Java]{ringMember.java}
\newpage
\addcontentsline{toc}{subsection}{4.4 \quad criticalSection.java}
\subsection*{criticalSection.java}
\lstinputlisting[language=Java]{criticalSection.java}
\newpage
\addcontentsline{toc}{subsection}{4.1 \quad TokenObject.java}
\subsection*{TokenObject.java}
\lstinputlisting[language=Java]{TokenObject.java}
\end{document}
